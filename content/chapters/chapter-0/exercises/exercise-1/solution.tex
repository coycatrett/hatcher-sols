<p>
    We may remove any point of our choosing from the torus. Thinking of the torus $\mathbb{T}$ as $\mathbb{S}^1 \times
    \mathbb{S}^1$, let $p_0 = (z_0,w_0) \in \mathbb{T}$.

    If $\widetilde{H} : (\mathbb{T} - p_0) \times I \to \mathbb{T}
    - p_0$ is a deformation retraction of $\mathbb{T} - p_0$ onto the union of a longitudinal circle (of the form
    $\mathbb{S}^1 \times \{z'\}$, around the $\textit{long}$ way) and a meridional circle
    (of the form $\{w'\} \times \mathbb{S}^1$), and $p_1 = (z_1, w_1)$ is any other point of $\mathbb{T} - p_0$,
    let $h : \mathbb{T} \to \mathbb{T}$ be the homeomorphism taking $p_1$ to $p_0$ defined by
    $$ h(z,w) = (z_0z_1^{-1}z, w_0w_1^{-1}w). $$

    The composite $h^{-1} \circ \widetilde{H} \circ (h \times \text{id}_I) : (\mathbb{T} - p_1)\times I \to \mathbb{T} -
    p_1$ is a deformation retraction of $\mathbb{T} - p_1$ to the union of a longitudinal circle $\mathbb{S}^1 \times
    \{w_0w_1^{-1}w'\}$ and meridional circle $ \{z_0z_1^{-1}z'\} \times \mathbb{S}^1$.
</p>

<p>
    With this out of the way and now thinking of $\mathbb{T}$ as the quotient
    $$[-1,1]^{2} /\big((-1,t) \sim (1,t) \text{ and } (t,-1) \sim (t,1) \forall t \in [-1,1]\big)$$
    with quotient map $\rho : [-1,1]^2 \to \mathbb{T}$, let $p_0 = \rho(0)$ be the point that we remove from
    $\mathbb{T}$.
</p>

<p>
    In our current representation of the torus, the longitude and meridian circles of the torus that we will retract to
    are the images of of $[-1,1] \times \{0,1\}$ and $\{0,1\} \times [-1,1]$ under $\rho$.

    These circles correspond to opposite pairs of edges in the boundary square $\partial [-1,1]^{2}$ and they intersect
    once at the image of the four corners of $[-1,1]^{2}$ under $\rho$.
</p>

<p>
    Given $x \in [-1,1]^{2}- 0$, put $r(x)$ to be the point on $\partial [-1,1]^{2}$ where the ray based at
    the origin in the direction of $x$ intersects $\partial [-1,1]^{2}$.
</p>

<div class="img-container">
    <img src="/static/images/0/1/radial-projection-from-origin-infinity-norm.png">
</div>

<p>
    The map $r : [-1,1]^{2} - 0 \to \partial [-1,1]^{2}$ above is the radial projection from the origin onto
    $\partial [-1,1]^{2}$ and has explicit formula given by
    $$r(x) = \frac{x}{\,\,\lVert x \rVert_{\infty}},$$
    where $\lVert \, \cdot \, \rVert_{\infty}$ is the
    <a
        href="https://en.wikipedia.org/wiki/Norm_(mathematics)#Maximum_norm_(special_case_of:_infinity_norm,_uniform_norm,_or_supremum_norm)">$\infty$-norm</a>.
</p>

<p>
    The significance of the $\infty$-norm above is that unit disk $\mathbb{D}^{2}_{\infty}$ in $\mathbb{R}^2$ with the
    $\infty$-norm is precisely $[-1,1]^{2}$, and the unit circle $\mathbb{S}^{1}_{\infty}$ in $\mathbb{R}^2$ with the
    $\infty$-norm is precisely $\partial[-1,1]^{2}$.
    Therefore, the above is simply an instance of the radial projection from the origin to the unit circle in a normed
    vector space. For these reasons, we will refer to $[-1,1]^{2}$ as $\mathbb{D}^{2}$ and $\partial [-1,1]^{2}$ as
    $\mathbb{S}^{1}$ for the remainder of the solution.
</p>

<p>
    Let $\iota : \mathbb{S}^1 \hookrightarrow \mathbb{D}^2 - 0$ be the standard inclusion. Using our previously defined
    notation, notice that $r : \mathbb{D}^2 - 0 \to \mathbb{S}^1$ satisfies $r \iota = \text{id}_{\mathbb{S}^1}$, and so
    $r$ is a retraction of $\mathbb{D}^2 - 0$ onto $\mathbb{S}^1$.
</p>

<p>
    Define the linear homotopy $H :(\mathbb{D}^{2} - 0) \times I \to \mathbb{D}^{2} - 0$ by
    $$H(-,t) = (1 - t)\text{id}_{\mathbb{D}^{2} - 0} + t\,\iota r.$$
    Since $H(x,-)$ parametrizes the line segment from $x$ to $r(x)$ for each $x$ (and is therefore never
    $0$), $H$ is well-defined. $H$ may be written as a composition of continuous functions, and is therefore
    continuous.
</p>

<p>One can now verify that</p>

<ul style="margin: 0; padding-top: 0;">
    <li>
        $H(x,0) = x$ for all $x \in \mathbb{D}^{2} - 0$
    </li>
    <li>
        $H(x,1) = \iota r(x) \in \mathbb{S}^{1}$ for all $x \in \mathbb{D}^{2} - 0$
    </li>
    <li>
        $H(x,t) = x$ for all $(x,t) \in \mathbb{S}^{1} \times I$
    </li>
</ul>

<p>
    and so $H$ is a deformation retraction of $\mathbb{D}^2 - 0$ onto $\mathbb{S}^{1}$.
</p>

<p>
    Recall that $p_{0} = \rho(0)$ is the point we are removing from $\mathbb{T}$. Let $\rho_{0} : \mathbb{D}^{2} - 0 \to
    \mathbb{T} - p_{0}$ be the restriction of $\rho$ to $\mathbb{D}^{2} - 0$. Since $I$ and $\mathbb{T}$ are
    locally compact spaces, then $\rho_0 \times \text{id}_{I} : (\mathbb{D}^2 - 0) \times I \to (\mathbb{T} - p_0)
    \times I$ is a quotient map (see <a href="https://math.stackexchange.com/a/31700/803927">this MSE post</a> for why
    local compactness is required).
</p>

<p>Notice that $\rho_0 \circ H$ is constant on the fibers of $\rho_0 \times
    \text{id}_{I}$, and therefore descends to a map $\widetilde{H} : (\mathbb{T} - p_0) \times I \to \mathbb{T} - p_0$
    by the universal property of the quotient space. In otherwords, the following square commutes:
</p>

<!-- https://q.uiver.app/#q=WzAsNCxbMCwwLCIoXFxtYXRoYmJ7RH1eMiAtIDApIFxcdGltZXMgSSAiXSxbMSwwLCJcXG1hdGhiYntEfV4yIC0gMCJdLFswLDEsIihcXG1hdGhiYntUfSAtIHBfMCkgXFx0aW1lcyBJIl0sWzEsMSwiXFxtYXRoYmJ7VH0gLSBwXzAiXSxbMCwxLCJIIl0sWzAsMiwiXFxyaG9fezB9IFxcdGltZXMgXFx0ZXh0e2lkfV97SX0iLDJdLFsyLDMsIlxcd2lkZXRpbGRle0h9IiwyLHsic3R5bGUiOnsiYm9keSI6eyJuYW1lIjoiZGFzaGVkIn19fV0sWzEsMywiXFxyaG9fMCJdXQ== -->
<iframe
    src="https://q.uiver.app/#q=WzAsNCxbMCwwLCIoXFxtYXRoYmJ7RH1eMiAtIDApIFxcdGltZXMgSSAiXSxbMSwwLCJcXG1hdGhiYntEfV4yIC0gMCJdLFswLDEsIihcXG1hdGhiYntUfSAtIHBfMCkgXFx0aW1lcyBJIl0sWzEsMSwiXFxtYXRoYmJ7VH0gLSBwXzAiXSxbMCwxLCJIIl0sWzAsMiwiXFxyaG9fezB9IFxcdGltZXMgXFx0ZXh0e2lkfV97SX0iLDJdLFsyLDMsIlxcd2lkZXRpbGRle0h9IiwyLHsic3R5bGUiOnsiYm9keSI6eyJuYW1lIjoiZGFzaGVkIn19fV0sWzEsMywiXFxyaG9fMCJdXQ==&embed"></iframe>

<p>
    Similar to before, one verifies that
</p>

<ul style="margin: 0; padding-top: 0;">
    <li>
        $\widetilde{H}(\rho_{0}(x),0) = \rho_{0}(x)$ for all $x \in \mathbb{D}^{2} - 0$
    </li>
    <li>
        $\widetilde{H}(\rho_{0}(x),1) = \rho_{0}(\iota r(x)) \in \rho_{0}(\mathbb{S}^{1})$ for all $x \in
        \mathbb{D}^{2} - 0$
    </li>
    <li>
        $\widetilde{H}(\rho_{0}(x),t) = \rho_{0}(x)$ for all $(x,t) \in \mathbb{S}^{1} \times I$
    </li>
</ul>
<p>
    and so $\widetilde{H}$ is a deformation retraction onto $\rho_{0}(\mathbb{S}^{1})$, i.e., the union of
    the longitude and meridian circles in $\mathbb{T}$.
</p>

<p>
    One can then show that $\widetilde{H}$ has explicit formula
    $$\widetilde{H}(\rho_{0}(x), t) = \rho_{0} \left((1-t)x + t\frac{x}{\,\,\lVert x
    \rVert_{\infty}}\right).$$

    Obtaining formulae for various representations of $\mathbb{T}$ from this one is easier now,
    as we need only pass through a homeomorphism.
</p>

<!-- TODO: Add discussion about other explicit formula for a retraction -->