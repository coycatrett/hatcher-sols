<p>
    We prove this statement in slightly more generality. Let $(V,\lVert \, \cdot \, \rVert)$ be a normed vector space
    over the
    real or complex numbers with unit circle $\mathbb{S} = \{v \in V : \lVert v \rVert = 1\}$.
</p>

<p>
    Given $x \in V - 0$, let $r(x)$ to be the point on the unit circle
    where the ray based at the origin in the direciton of $x$ intersects $\mathbb{S}$,
    as indicated below.
</p>

<div class="img-container">
    <img src="/static/images/0/2/radial-projection-from-origin.png" >
</div>

<p>
    The map $r : V - 0 \to \mathbb{S}$ above is the radial projection from the origin
    onto the unit circle and has explicit formula given by
    $$r(x) = \frac{x}{\lVert x \rVert}.$$
    Note that $r$ is well-defined since $\lVert x \rVert$ is never zero on $\mathbb{R}^{n} - 0$.
    One can see that $r$ is continuous since it is the composition of continuous maps. If $\iota : \mathbb{S}
    \hookrightarrow V - 0$ is the standard inclusion map, one sees that $r$ is a retraction since $r\iota =
    \text{id}_{\mathbb{S}^1}$.
</p>

<p>
    Define the linear homotopy $H : (V - 0) \times I \to V - 0$ by
    $$H(-,t) = (1 - t)\text{id}_{V - 0} + t\,\iota r.$$
    Since $H(x,-)$ parametrizes the line segment from $x$ to $r(x)$ for each $x$ (and so is never $0$), then
    $H$ is well-defined. $H$ can be written as a composition of continuous functions, and is therefore
    continuous.
</p>

<p>
    One can now verify that
</p>

<ul style="margin: 0; padding-top: 0;">
    <li>
        $H(x,0) = x$ for all $x \in V - 0$
    </li>
    <li>
        $H(x,1) = \iota r(x) \in \mathbb{S}$ for all $x \in V - 0$
    </li>
    <li>
        $H(x,t) = x$ for all $(x,t) \in \mathbb{S} \times I$
    </li>
</ul>

<p>
    and so $H$ is a deformation retraction of $V - 0$ onto the unit circle $\mathbb{S}$. Specializing this argument to
    $V = \mathbb{R}^n$ with the usual norm on $\mathbb{R}^n$ gives an explicit deformation retraction
    $$H(x,t) = (1-t)x + t\frac{x}{\lVert x \rVert} = \left((1-t)x_1 + t\frac{x_1}{\sqrt{\sum_{i =
    1}^nx_i^2}},\dots,(1-t)x_n + t\frac{x_n}{\sqrt{\sum_{i = 1}^nx_i^2}}\right)$$
</p>